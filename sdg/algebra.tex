{
\newcommand{\EE}{\mathbf{E}}
\newcommand{\ev}{\operatorname{\mathrm{ev}}}
\newcommand{\Trunc}[2][]{\lVert{#2}\rVert_{#1}}
\newcommand{\quotient}[2]{{#1}/{#2}}

\subsection{Endomorphism Theories}

Let \(X\) be a type and \(n\) be a natural number.
In the following, we will write \(\EE_X(n)\) for the type of functions \(X^n\to X\) and call its inhabitants \emph{\(n\)-ary operations}.
It contains \(n\) canonical operations, the projections \(\pi^n_1,\dots,\pi^n_n : \EE_X(n)\).
Given any type \(Z\), the projection functions induce an equivalence
\[
  (Z \to X^n) \to (Z\to X)^n, \quad h \mapsto (\pi^n_1\circ h,\dots,\pi^n_n\circ h).
\]
We make implicit use of this equivalence by identifying an \(n\)-tuple of functions \(h_1,\dots,h_n:Z\to X\) with the unique induced function \(h : Z\to X^n\) which satisfies \(\pi^n_i \circ h = h_i\).

\begin{definition}
  Let \(X\) be a set.
  An \emph{\(\EE_X\)-algebra} consists of a set \(A\) and a function of type
  \[
    \interpret{-}\colonequiv\interpret{-}_A : \prod_{\{n:\N\}}\prod_{f:\EE_X(n)} A^n \to A
  \]
  such that \(\interpret{-}\) is compatible with projections, in that
  % \begin{align*}
  %   \prod_{n:\N}\prod_{i:\{1,\dots,n\}} \interpret{\pi^n_i} \underset{A^n \to A}{=} \pi^n_i
  % \end{align*}
  for each \(n:\N\) and each \(i : \{1,\dots,n\}\)
  \[\interpret{\pi^n_i} \underset{A^n \to A}{=} \pi^n_i,\]
  and with composition, in that for all \(m,n:\N\), \(f:\EE_X(n)\) and \(g_1,\dots,g_n : \EE_X(m)\)
  \[\interpret{f\circ (g_1,\dots,g_n)} \underset{A^m \to A}{=} \interpret{f}\circ (\interpret{g_1},\dots,\interpret{g_n}).\]
  The set \(A\) is called the \emph{carrier} of the \(\EE_X\)-algebra and the function \(\interpret{-}\) the \emph{interpretation function}.
  A \emph{homomorphism of \(\EE_X\)-algebras} is a map between the underlying carriers which commutes with the respective interpretation functions.
  Concretely, a map \(\varphi : A \to B\) is a homomorphism if
  \[
    \varphi(\interpret{f}_A(a_1,\dots,a_n)) = \interpret{f}_B(\varphi(a_1),\dots,\varphi(a_n))
  \]
  for each \(f:\EE_X(n)\) and \(a_1,\dots,a_n : A\).
  Given two \(\EE_X\)-algebras \(A\) and \(B\), we write \(\Alg{\EE_X}(A,B)\) for the subset of maps \(A\to B\) which are morphisms of \(\EE_X\)-algebras.
  Here, we identify an algebra with its carrier, suppressing the interpretation function and compatibility equalities, as usual.
\end{definition}

\begin{remark}
  The \(\N\)-indexed family of sets \(\EE_X(n)\) together with the projection operations and the composition functions \(\EE_X(m)^n \to \EE_X(n) \to \EE_X(m)\) is also called the \emph{endomorphism clone} of the set \(X\).
  This is an instance of a more general concept called an \emph{abstract clone}, which allows for an invariant presentation of an algebraic theory.
  The theory \(\EE_X\) presented in our case is called the \emph{endomorphism theory} of \(X\).
\end{remark}

We may collect (small) \(\EE_X\)-algebras together with their morphisms into a category \(\Alg{\EE_X}\).
In the following, we will investigate some universal constructions in this category.

\paragraph{Direct Products}

Let \(I\) be a type and let \(A(i)\) be an \(I\)-indexed family of \(\EE_X\)-algebras.
There is a unique structure of an \(\EE_X\)-algebra on the dependent product \(A\colonequiv \prod_{i:I}A(i)\) which makes the evaluation maps \(A \to A(i), s \mapsto s(i)\) homomorphisms of \(\EE_X\)-algebras.
Namely, for all \(f:\EE_X(n)\), \(s_1,\dots,s_n:A\) and \(i:I\) we are forced to define
\[
  \interpret{f}_A(s_1,\dots,s_n)(i) \colonequiv \interpret{f}_{A(i)}(s_1(i),\dots,s_n(i)),
\]
which is obviously compatible with projections and composition and determines \(\interpret{f}_A\) uniquely.
It is easy to see that \(A\), together with the evaluation maps, satisfies the universal property of the \(\Trunc[0]{I}\)-indexed product in the category \(\Alg{\EE_X}\).

% \begin{example}
%   The set \(X\) itself carries the structure of an \(\EE_X\)-algebra in a canonical fashion.
%   More generally, for a type \(Y\), the set of functions \(Y \to X\) carries the structure of an \(\EE_X\)-algebra by setting
%   \[\interpret{f}_{Y\to X} : (Y\to X)^n \to (Y\to X), \quad (s_1,\dots,s_n) \mapsto f\circ (s_1,\dots,s_n)\]
%   for \(f:\EE_X(n)\), which is easily seen to be compatible with projections and composition.
%   % Moreover, we write \(X_i\colonequiv\pi^n_i : F(n)\) for the \(n\) projection functions
%   % We call this the \emph{free \(\EE_X\)-algebra} on \(n\) generators.
%   % More specifically, the generators are given by the projection functions \(\pi^n_1,\dots,\pi^n_n : F(n)\)
% \end{example}

\paragraph{Finitely Generated Free Algebras}
As a special case of the previous construction, every function space into an \(\EE_X\)-algebra can be equipped with a canonical \(\EE_X\)-algebra structure.
Since \(X\) itself is an \(\EE_X\)-algebra in a canonical fashion, this applies in particular for each of the sets \(\EE_X(m)\equiv X^m\to X\), where \(m:\N\).
To distinguish this \(\EE_X\)-algebra from the set \(\EE_X(m)\) of \(m\)-ary operations, we denote it by \(F(m)\).
Moreover, we write \(X_1\colonequiv \pi^m_1, \dots X_m\colonequiv\pi^m_m : F(m)\) for the projection functions.
The following lemma shows that \(F(m)\) is the \emph{free \(\EE_X\)-algebra} generated by \(m\) generators \(X_1,\dots,X_m\).

\begin{lemma}
  Let \(m\) be a natural number.
  For every \(\EE_X\)-algebra \(A\), the canonical map
  \[
    \Alg{\EE_X}(F(m),A) \to A^m, \varphi \mapsto (\varphi(X_1),\dots,\varphi(X_m))
  \]
  is an equivalence.
\end{lemma}
\begin{proof}
  Given an \(m\)-tuple \(\vec a\equiv (a_1,\dots,a_m) : A^n\) we define a map \(\ev_{\vec a} : F(m)\to A\) by setting
  \[\ev_{\vec a}(g) \colonequiv \interpret{g}_A(\vec a)\]
  for \(g:F(m)\).
  This is in fact a morphism of \(\EE_X\)-algebras.
  Namely, for every \(n:\N\), \(f:\EE_X(n)\) and \(g_1,\dots,g_n : F(m)\) we compute
  \begin{align*}
    \interpret{f}_A(\ev_{\vec a}(g_1),\dots,\ev_{\vec a}(g_n)) & = \interpret{f}_A(\interpret{g_1}_A(\vec a),\dots,\interpret{g_n}_A(\vec a)) \\
    & = \interpret{f\circ(g_1,\dots,g_n)}_A (\vec a)\\
    & = \interpret{\interpret{f}_{X^m\to X}(g_1,\dots,g_n)}_A (\vec a)\\
    & = \ev_{\vec a}(\interpret{f}_{X^m \to X}(g_1,\dots,g_n)).
  \end{align*}
  Note that \(\ev_{\vec a}(X_i) = a_i\), so \(\ev_{\vec a}\) lies in the fiber over \(\vec a\) of the map in question.
  Conversely, given a morphism \(\varphi : F(m)\to A\) of \(\EE_X\)-algebras, set \(a_i\colonequiv \varphi(X_i)\).
  Now, given any \(g:F(m)\), we have
  \begin{equation*}
    \ev_{\vec a}(g)
    \equiv \interpret{g}_A(\varphi(X_1),\dots,\varphi(X_m))
     = \varphi(\interpret{g}_{F(m)}(X_1,\dots,X_m))
     \equiv \varphi(g\circ (X_1,\dots,X_m))
     = \varphi(g\circ\id) \equiv \varphi(g).
  \end{equation*}
  This shows that the mapping \(\vec a \mapsto \ev_{\vec a}\) defines an inverse to the map in question.
  % \begin{align*}
  %   \ev_{\vec a}(g) & \equiv \interpret{g}_A(\varphi(X_1),\dots,\varphi(X_m))\\
  %   & = \varphi(\interpret{g}_{F(m)}(X_1,\dots,X_m)) \\ 
  %   & \equiv \varphi(g\circ (X_1,\dots,X_m)) \\
  %   & = \varphi(g).
  % \end{align*}
\end{proof}

\paragraph{Quotients}

Recall that an equivalence relation on a type \(A\) is a proposition-valued binary relation \(\sim\) on \(A\) which is reflexive, symmetric and transitive.
In case \(A\) is the underlying carrier of an \(\EE_X\)-algebra we call \(\sim\) a \emph{congruence relation} if for every \(n\)-ary operation \(f:\EE_X(n)\) and all \(a_1,\dots,a_n,b_1,\dots,b_n : A\) the following implication holds:
\[
  a_1 \sim b_1 \times \dots\times a_n \sim b_n \to \interpret{f}(a_1,\dots,a_n) \sim \interpret{f}(b_1,\dots,b_n).
\]
If \(\quotient{A}{\sim}\) is the set quotient of \(A\) by \(\sim\) with quotient map \([-]:A \to \quotient{A}{\sim}\), there is a unique \(\EE_X\)-algebra structure on \(\quotient{A}{\sim}\) such that \([-]\) becomes a morphism of \(\EE_X\)-algebras.
Namely, for \(f:\EE_X(n)\) and \(a_1,\dots,a_n:A\) we have to set
\[
  \interpret{f}_{\quotient{A}{\sim}}([a_1],\dots,[a_n]) \colonequiv [\interpret{f}_A(a_1,\dots,a_n)],
\]
which is well-defined as \(\sim\) is a congruence relation, is compatible with projections and composition, and determines \(\interpret{f}_{\quotient{A}{\sim}}\) uniquely.


\subsection{Ringed Endomorphism Theories}

We will now assume that our base object \(X\equiv R\) carries the structure of a ring.%
\footnote{As usual, we assume all rings to be commutative.}
\begin{lemma}
  Every \(\EE_R\)-algebra carries the structure of an \(R\)-algebra in a canonical fashion and every homomorphism of \(\EE_R\)-algebras is a homomorphism of \(R\)-algebras with respect to this structure.
\end{lemma}
\begin{proof}
  Let \(A\) be an \(\EE_R\)-algebra.
  We obtain the structure of a ring on \(A\) by interpreting the ring operations \(0,1:1\to R\), \(- : R\to R\) and \(+,\cdot : R^2\to R\) using the interpretation function of \(A\).
  Note that since it is a function, it preserves equalities, and thus \(A\) satisfies the ring axioms because \(R\) does.
  In particular, every homomorphism of \(\EE_R\)-algebras preserves this canonical ring structure.
  Note that since \(R = (1\to R)= F(0)\) is the \emph{initial} \(\EE_R\)-algebra, every \(\EE_R\)-algebra \(A\) admits a unique homomorphism of \(\EE_R\)-algebras \(R\to A\), giving it the structure of an \(\R\)-algebra.
  By initiality, every morphism of \(\EE_R\)-algebras commutes with this unique morphism from \(R\), making it a homomorphism of \(\R\)-algebras.
\end{proof}

Recall that a congruence relation on a ring \(A\) is an equivalence relation which is compatible with the ring operations \(-\), \(+\), \(\cdot\) in an obvious manner.
It is easy to see that each such congruence relation gives rise to an ideal \(I_\sim\colonequiv \{a:A\mid a\sim 0\}\) of \(A\) and that, conversely, every ideal \(I\) determines a congruence relation \(\sim_I\) on \(A\) by setting \(a\sim_I b\colonequiv a-b\in I\).
Moreover, these constructions are mutually inverse to each other.

Now suppose that \(A\) is an \(\EE_R\)-algebra.
Every congruence relation on \(A\) with respect to the \(\EE_R\)-algebra structure is in particular a congruence relation with respect to the underlying ring structure and thus determines an ideal.
However, it may no longer be the case that every ideal gives rise to an \(\EE_R\)-congruence relation, as the following example shows.

\begin{example}
  \rednote{Missing.}
\end{example}

We will show in proposition ?? that the correspondence between ideals and congruence relations for \(\EE_R\)-algebras can be restored by restricting to rings \(R\) which are \emph{Fermat-Hadamard rings} in the sense of \cref{def:hadamard-quotient} below.
To make this definition more readable, let us introduce some notational conventions.
Recall that we denoted the projection operations \(R^n\to R\) by \(X_1,\dots, X_n\).
By suppressing the composition symbol \(\circ\), we may write every \(n\)-ary operation \(f:\EE_X(n)\) as
\[
  f = f\circ\id_{X^n} = f \circ (X_1,\dots,X_n) = f(X_1,\dots,X_n),
\]
allowing us to treat the projection operations as variables.
We will also use different variable names like \(X,Y,Z_1,\dots,Z_n\).
Moreover, the ring operations will be written in infix notation as usual.
Lastly, we will make use of vector notation for variables, for example, we have \(f = f(\vec X)\).


\begin{definition}
  \label{def:hadamard-quotient}
  Given an \((n+1)\)-ary operation \(f:\EE_R(n+1)\), a \emph{Hadamard quotient} of \(f\) is an \((n+2)\)-ary operation \(g : \EE_R(n+2)\) which satisfies
  \[
    f(X,\vec Z) - f(Y,\vec Z) = (X-Y)\cdot g(X,Y,\vec Z) \qquad \text{(in \(\EE_R(n+2)\))},
  \]
  where we denote the projection operations in \(\EE_R(n+2)\) by \(X,Y,Z_1,\dots,Z_n\).
  The ring \(R\) is said to be a \emph{Fermat-Hadamard ring} if each \((n+1)\)-ary operation \(f:\EE_R(n+1)\) admits a unique Hadamard quotient.
\end{definition}

\begin{example}
  Working within the axiom system of synthetic algebraic geometry \cite{draft}, the base ring \(R\) is a Fermat-Hadamard ring.
  This is because the duality axiom implies that \(\EE_R(n) = R^n \to R = R[X_1, \ldots, X_n]\) for all \(n : \N\).
  That is, every function is a polynomial.
  Polynomial functions preserve ideal membership so existence of a Hadamard quotient follows immediately from \ref{prop:ideals-congruence-fermat-ring}.
  The uniqueness follows from the fact that \(X_1 - X_2\) is a regular element in any polynomial ring.
\end{example}

\begin{example}
  In the Dubuc topos \(\mathrm{Dub}\), the smooth reals \(R = \yo \ell C^\infty(\R)\) foorm a Fermat-Hadamard ring.
  We use the Kripke-Joyal semantics to show that 
  \[ \mathrm{Dub} \models \forall f : R^{n+1} \to R \; 
                          \exists ! \; g : R^{n+2} \to R  \;
                          \forall x, y , \bar{z}  : R \; \;
                           f(x, \bar{z}) - f(y, \bar{z}) = (x-y)g(x,y,\bar{z})\]
  Unwinding the Kripke-Joyal semantics, and applying the Yoneda lemma this statement is equivalent to the following:
  For all germ determined \(C^\infty\)-rings \(C^\infty(\mathbb{R}^m)/I\), and for all \(f : C^{\infty}(\mathbb{R}^{m+n+1})/I\) there are smooth \(u_{\alpha}: \mathbb{R}^{m}\to \mathbb{R}\), summing to one, 
  and \(g_{\alpha}: C^\infty(\mathbb{R}^{m+n+3})/(I, u_{\alpha}X_{m+1}-1)\) such that for all \(x,y,\bar z : C^\infty(\mathbb{R}^{m+1})/(I, u_{\alpha}X_{m+1}-1)\) we have 
  \[f_\alpha(x, \bar{z}) - f_\alpha(y, \bar{z}) = (x-y)g_\alpha(x,y,\bar{z})\] 
  where \(f_\alpha\) is \(f\) included in the quotient ring.

We show this. Given an \(f : C^\infty(\mathbb{R}^{m+n+1})/I\), we choose the \(u_\alpha\) to be trivial.
By Hadamard's lemma, for each \(a : \mathbb{R}^n\) there is a function \(g(a)\) such that for all \(x, y, \bar{z} : \mathbb{R}\) 
\[f(a)(x, \bar{z}) - f(a)(y, \bar{z}) = (x-y)g(a)(x,y,\bar{z})\]
This determines a function \(g : \mathbb{R}^{m+n+2} \to \mathbb{R}\), which satisfies the desired criteria when descending to the quotient.

This is unique by regularity of \(X-Y\) as an element of \[C^\infty(\mathbb{R}^{m+n+2})/I = C^\infty(\mathbb{R}^m)/I \otimes C^\infty(\mathbb{R}^{n+2})\]
If $g, g'$ both satisfy the condition, then $(X-Y)(g-g') = 0$, and so $g = g'$.
\end{example}

\begin{proposition}
  \label{prop:fermat-ring-quotients}
  \rednote{Conjecture}
  Let \(R\) be a ring.
  The following are equivalent:

  \begin{enumerate}
    \item
      For every ideal \(I\) of every \(\EE_R\)-algebra \(A\), the equivalence relation \(\sim_I\) is a congruence relation.
    \item
      For every ideal \(I\) of every finitely generated free \(\EE_R\)-algebra, the equivalence relation \(\sim_I\) is a congruence relation.
    \item
      The ring \(R\) is a Fermat-Hadamard ring.
  \end{enumerate}
\end{proposition}
\begin{proof}
  The first implication is trivial.
  Thus, suppose that the second proposition holds and let \(f:\EE_R(n+1)\) be an arbitrary operation.
  Consider the ideal \(I\) generated by \(X-Y\) in the free algebra \(F(n+2)\) on \(n+2\) generators \(X,Y,Z_1,\dots,Z_n\).
  To construct a Hadamard quotient of \(f\) it is enough to show that \(f(X,\vec Z)-f(Y,\vec Z)\) lies in the ideal \(I\), i.e.~that \(f(X,\vec Z)\sim_I f(Y,\vec Z)\) holds.
  By definition of the \(\EE_X\)-algebra structure on \(F(n+2)\) we have
  \[
    f(X,\vec Z) = f \circ (X,\vec Z) = \interpret{f}_{F(n+2)}(X,\vec Z)
  \]
  and similarly for \(f(Y,\vec Z)\).
  Therefore, since \(\sim_I\) is a congruence by assumption, it suffices to show that \(X\sim_I Y\) and \(Z_1 \sim_I Z_1\), \dots, \(Z_n\sim_I Z_n\), which is clearly the case.
  \rednote{For uniqueness of the Hadamard quotient, we need to show that \(X-Y\) is a regular element in \(F(n+2)\).
    This holds for Fermat theories, see [Dubuc-Kock, paragraph before Prop. 1.1].
    Maybe add locality condition?}

  \rednote{\(3\Rightarrow 1\) is in Dubuc-Kock}
\end{proof}



TODO:
\begin{enumerate}
  \item Localization?
\end{enumerate}

}
